\documentclass{article}
\usepackage{graphicx} % Required for inserting images
\usepackage{kotex} % 한글 사용을 위한 패키지
\usepackage{amsmath}
\usepackage{amsfonts}

\title{논문 이해를 돕는 추가자료}
\author{Mose Park}
\date{June 2024}

\begin{document}

\maketitle

\section{행렬의 대각화 가능성 예제 문제}

주어진 행렬 \(A\):
\[
A = \begin{pmatrix}
4 & 1 \\
2 & 3
\end{pmatrix}
\]

이 행렬이 대각화 가능한지 확인하고, 만약 가능하다면 대각화 행렬을 구해보기

\subsection{고유값(Eigenvalue) 계산}


   행렬 \(A\)의 고유값을 구하기 위해 특성 방정식 풀기:
   \[
   \det(A - \lambda I) = 0
   \]
   여기서 \(I\)는 단위행렬이고, \(\lambda\)는 고유값. 먼저 \(A - \lambda I\)를 계산:
   \[
   A - \lambda I = \begin{pmatrix}
   4 - \lambda & 1 \\
   2 & 3 - \lambda
   \end{pmatrix}
   \]
   이 행렬의 행렬식을 계산:
   \[
   \det(A - \lambda I) = (4 - \lambda)(3 - \lambda) - (2 \cdot 1) = \lambda^2 - 7\lambda + 10
   \]
   특성 방정식을 풀면:
   \[
   \lambda^2 - 7\lambda + 10 = 0
   \]
   이 방정식을 인수분해하면:
   \[
   (\lambda - 5)(\lambda - 2) = 0
   \]
   따라서, 고유값은 \(\lambda_1 = 5\)와 \(\lambda_2 = 2\)입니다.

\subsection{고유벡터(Eigenvector) 계산}


   각 고유값에 대응하는 고유벡터 구하기

   - \(\lambda_1 = 5\):
     \[
     (A - 5I) \mathbf{v}_1 = 0 \Rightarrow \begin{pmatrix}
     -1 & 1 \\
     2 & -2
     \end{pmatrix} \mathbf{v}_1 = \mathbf{0}
     \]
     이 방정식을 풀면:
     \[
     \mathbf{v}_1 = \begin{pmatrix} 1 \\ 1 \end{pmatrix}
     \]

   - \(\lambda_2 = 2\):
     \[
     (A - 2I) \mathbf{v}_2 = 0 \Rightarrow \begin{pmatrix}
     2 & 1 \\
     2 & 1
     \end{pmatrix} \mathbf{v}_2 = \mathbf{0}
     \]
     이 방정식을 풀면:
     \[
     \mathbf{v}_2 = \begin{pmatrix} -1 \\ 2 \end{pmatrix}
     \]

\subsection{대각화 행렬 구하기}


   고유벡터들을 열벡터로 하는 행렬 \(P\)를 구성:
   \[
   P = \begin{pmatrix}
   1 & -1 \\
   1 & 2
   \end{pmatrix}
   \]
   대각행렬 \(D\)는 고유값들을 대각 성분으로 갖는 행렬:
   \[
   D = \begin{pmatrix}
   5 & 0 \\
   0 & 2
   \end{pmatrix}
   \]

   이제 \(A\)를 대각화하는지를 확인하기 위해 \(P^{-1}AP\)를 계산. 먼저 \(P\)의 역행렬 \(P^{-1}\)를 구하기:
   \[
   P^{-1} = \frac{1}{\det(P)} \text{adj}(P) = \frac{1}{3} \begin{pmatrix}
   2 & 1 \\
   -1 & 1
   \end{pmatrix}
   \]
   여기서 \(\det(P) = 3\). 이제 \(P^{-1}AP\)를 계산하면:
   \[
   P^{-1}AP = \begin{pmatrix}
   2/3 & 1/3 \\
   -1/3 & 1/3
   \end{pmatrix} \begin{pmatrix}
   4 & 1 \\
   2 & 3
   \end{pmatrix} \begin{pmatrix}
   1 & -1 \\
   1 & 2
   \end{pmatrix} = \begin{pmatrix}
   5 & 0 \\
   0 & 2
   \end{pmatrix} = D
   \]

따라서, 행렬 \(A\)는 대각화 가능하며, 대각 행렬은 \(P\)와 \(D\)입니다. \(A = PDP^{-1}\)의 형태로 나타낼 수 있습니다.

\end{document}
