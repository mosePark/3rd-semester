\documentclass[10pt, aspectratio=169]{beamer}

\usetheme[progressbar=frametitle]{metropolis}
\usecolortheme{aggie}

\usepackage{appendixnumberbeamer}
\usepackage{tcolorbox} % 내 설정
\tcbuselibrary{breakable}
\usepackage{booktabs}
\usepackage[scale=2]{ccicons}

\usepackage{amsmath}
\usepackage{amsfonts}
\usepackage{enumitem}
\usepackage{ltablex}
\usepackage{amssymb}
\usepackage{graphicx}
\usepackage{pgfplots}
\usepackage{booktabs}
\usepackage{array}
\usepackage{tabularx}

\usepgfplotslibrary{dateplot}

\usepackage{xspace}
\newcommand{\themename}{\textbf{\textsc{metropolis}}\xspace}
\newcommand{\printfnsymbol}[1]{%
  \textsuperscript{\@fnsymbol{#1}}%
}
\makeatother
\newenvironment{callig}{\fontfamily{qcr}\selectfont}{}
\newcommand{\textcallig}[1]{{\callig#1}}
\newcommand{\f}{v}
\newcommand{\g}{g}
\newcommand{\hY}{\hat{y}}
\newcommand{\x}{x}
\newcommand{\z}{z}
\newcommand{\I}{\delta}
%\newcommand{\R}{\mathbb{R}}
\newcommand{\expd}{\text{\textcallig{p}}}
\newcommand{\ex}{\Expl}
\newcommand{\prodd}{\Upsilon}
\def\Expl{\mathcal{E}}
\newcommand{\e}{\mathbf{e}}
\newcommand{\m}{m}
\newcommand{\M}{M}
\newcommand{\bfalpha}{\mathbf{\alpha}}
\newcommand{\veck}{\mathbf{k}}
\newcommand*{\medcup}{\mathbin{\scalebox{1.5}{\ensuremath{\cup}}}}
\newcommand{\mubar}[1]{\bar{\mu}_{#1}}
\newcommand{\setlessell}{\mathcal{S}_\ell }
\newcommand{\bfb}{\mathbf{b} }
\DeclareMathOperator{\spann}{span}
\DeclareMathOperator{\calA}{\mathcal{A}}
\newcommand*{\defeq}{\stackrel{\text{def}}{=}}
\DeclareMathOperator*{\argmax}{arg\,max}
\DeclareMathOperator*{\argmin}{arg\,min}
\newcommand{\norm}[1]{\left\lVert#1\right\rVert}



% start == Title Page %%%
\title{Multiple Testing}
\subtitle{In Nonparametric Hidden Markov Models:
 An Empirical Bayes Approach}
% \date{\today}
\date{June 4, 2024}
\author{mose Park}
\institute{Department of Statistical Data Science \\
    University of Seoul}
\vfuzz=20pt
\hfuzz=10pt
% end == Title Page %%%

%%% box
% 사용자 정의 tcolorbox 환경을 만듭니다.
\newtcolorbox{myaxiombox}[2][]{%
    title=#2,
    colback=white,
    colframe=red!50!black,
    coltitle=white,
    fonttitle=\bfseries,
    rounded corners,
    boxsep=5pt,
    boxrule=1pt,
    #1 % 추가 옵션을 위한 공간
}

\newtcolorbox{mydefbox}[2][]{%
    title=#2,
    colback=white,
    colframe=red!50!gray, % 여기에 오타 수정
    coltitle=white,
    fonttitle=\bfseries,
    rounded corners,
    boxsep=5pt,
    boxrule=1pt,
    #1 % 추가 옵션을 위한 공간
}

\newtcolorbox{mytheorembox}[2][]{%
    title=#2,
    colback=white,
    colframe=red!50!gray, % 여기에 오타 수정
    coltitle=white,
    fonttitle=\bfseries,
    rounded corners,
    boxsep=5pt,
    boxrule=1pt,
    #1 % 추가 옵션을 위한 공간
}

%%%
\begin{document}

\maketitle

%===
\begin{frame}{Table of contents}
  \setbeamertemplate{section in toc}[sections numbered]
  \tableofcontents%[hideallsubsections]
\end{frame}
%===
\section[Intro]{Introduction}
%===
%===
\begin{frame}{Aim of the paper}
    \begin{figure}[h]
    \centering
    \includegraphics[width=1.0\textwidth]{fig-2/aim.png}
    \end{figure}

    \begin{itemize}[label=\scalebox{0.5}{$\blacksquare$}]
        \vspace{1.5em}
        \itemsep1.2em
        \item $ "\theta = 0" $ means \textbf{typical disease} levels. → ex) \textit{A common cold}
        \item $ "\theta = 1" $ means \textbf{atypical outbreak}. → ex) \textit{MERS, Covid-19}
    \end{itemize}
\end{frame}
% 데이터 X, 마코프 체인으로부터 추출된 관찰되지 않은 범주형 변수 theta (0,1)
% 독립성 기반 가정 고려 X
% hidden state theta로부터 생성된 obs data X 
%===
\begin{frame}
    
\end{frame}
%===
%===
\begin{frame}
    
\end{frame}
%===
%===
\begin{frame}
    
\end{frame}
%===
%===
\begin{frame}
    
\end{frame}
%===
%===
\begin{frame}
    
\end{frame}
%===
%===
\begin{frame}
    
\end{frame}
%===
%===
\begin{frame}
    
\end{frame}
%===
%===
\begin{frame}
    
\end{frame}
%===
%===
\begin{frame}
    
\end{frame}
%===
%===
\begin{frame}
    
\end{frame}
%===





%===
\end{document}
