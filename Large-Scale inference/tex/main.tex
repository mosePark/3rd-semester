% ref link : https://ko.overleaf.com/latex/templates/texas-a-and-m-university-metropolis-beamer-theme/spvwhrxxbyqb
% This presentation uses a template designed by Dan Drennan, licensed under CC BY 4.0.


\documentclass[10pt]{beamer}

\usetheme[progressbar=frametitle]{metropolis}
\usecolortheme{aggie}

\usepackage{appendixnumberbeamer}
\usepackage{tcolorbox} % 내 설정
\usepackage{booktabs}
\usepackage[scale=2]{ccicons}

\usepackage{pgfplots}
\usepgfplotslibrary{dateplot}

\usepackage{xspace}
\newcommand{\themename}{\textbf{\textsc{metropolis}}\xspace}
\newcommand{\printfnsymbol}[1]{%
  \textsuperscript{\@fnsymbol{#1}}%
}
\makeatother
\newenvironment{callig}{\fontfamily{qcr}\selectfont}{}
\newcommand{\textcallig}[1]{{\callig#1}}
\newcommand{\f}{v}
\newcommand{\g}{g}
\newcommand{\hY}{\hat{y}}
\newcommand{\x}{x}
\newcommand{\z}{z}
\newcommand{\I}{\delta}
%\newcommand{\R}{\mathbb{R}}
\newcommand{\expd}{\text{\textcallig{p}}}
\newcommand{\ex}{\Expl}
\newcommand{\prodd}{\Upsilon}
\def\Expl{\mathcal{E}}
\newcommand{\e}{\mathbf{e}}
\newcommand{\m}{m}
\newcommand{\M}{M}
\newcommand{\bfalpha}{\mathbf{\alpha}}
\newcommand{\veck}{\mathbf{k}}
\newcommand*{\medcup}{\mathbin{\scalebox{1.5}{\ensuremath{\cup}}}}
\newcommand{\mubar}[1]{\bar{\mu}_{#1}}
\newcommand{\setlessell}{\mathcal{S}_\ell }
\newcommand{\bfb}{\mathbf{b} }
\DeclareMathOperator{\spann}{span}
\DeclareMathOperator{\calA}{\mathcal{A}}
\newcommand*{\defeq}{\stackrel{\text{def}}{=}}
\DeclareMathOperator*{\argmax}{arg\,max}
\DeclareMathOperator*{\argmin}{arg\,min}
\newcommand{\norm}[1]{\left\lVert#1\right\rVert}


% start == Title Page %%%
\title{Faith-SHAP}
\subtitle{Faith-Shap: The Faithful Shapley Interaction Index}
% \date{\today}
\date{May 14, 2024}
\author{mose Park}
\institute{    Department of Statistical Data Science \\
    University of Seoul}
% \titlegraphic{\hfill\includegraphics[height=1.5cm]{logo.pdf}}
% end == Title Page %%%

%%% def axiom box
% 사용자 정의 tcolorbox 환경을 만듭니다.
\newtcolorbox{myaxiombox}[2][]{%
    title=#2,
    colback=white,
    colframe=red!50!black,
    coltitle=white,
    fonttitle=\bfseries,
    rounded corners,
    boxsep=5pt,
    boxrule=1pt,
    #1 % 추가 옵션을 위한 공간
}


%%%
\begin{document}

\maketitle

%===
\begin{frame}{Table of contents}
  \setbeamertemplate{section in toc}[sections numbered]
  \tableofcontents%[hideallsubsections]
\end{frame}
%===


\section[Intro]{Introduction}


%===
\begin{frame}{Coalition game}

\begin{figure}[h]
\centering
\includegraphics[width=0.8\textwidth]{fig/coalition-game.png}
\caption{coalition game}
\label{fig:coalition-game}
\end{figure}
  
\end{frame}


\def\model{f}
\def\X{\mathcal{X}}
\def\R{\mathbb{R}}
%===
\begin{frame}{Notation}

\begin{itemize}
    \item $\model : \X \mapsto \mathbb{R}$ with $\X \subseteq \R^d$ : A black-box model
    \vspace{2mm}
    \item $\f_{\x} : 2^d \rightarrow \mathbb{R}$ : A set function
    \vspace{2mm}
    \item $\ell \in [d]$ : The importance of interactions between features
    \vspace{2mm}
    \item $\ex$ : The importance function
    \vspace{2mm}
    \item $\ex(\f, \ell) = (\ex_{S}(\f, \ell))_{S \in \mathcal{S}_\ell} \in \R^{d_\ell}$
    \vspace{2mm}
    \begin{itemize}
      \item $\mathcal{S}_\ell$: Each coalition $S \subseteq [d]$ where $0 \leq |S| \leq \ell$
      \vspace{2mm}
      \item $d_{\ell} \defeq \sum_{j=0}^{\ell} { d \choose j}$: The number of possible coalitions at level $\ell$
      \vspace{2mm}
      \item $\ex(\f, \ell) \in \R^{d_\ell}$: The importance quantity at level $\ell$
    \end{itemize}
\end{itemize}

\end{frame}
%===
\section[Axioms]{Axioms}

\begin{frame}{Axioms}
    \begin{myaxiombox}{Interaction Linearity}
    For any maximum interaction order $\ell \in [d]$, and for any two set functions $\f_1$ and $\f_2$, and any two scalars $\alpha_1, \alpha_2 \in \mathbb{R}$, the interaction index satisfies: $\ex(\alpha_1 \f_1+ \alpha_2 \f_2,\ell) = \alpha_1 \ex(\f_1,\ell) + \alpha_2 \ex(\f_2,\ell)$.
    \end{myaxiombox}
\end{frame}
%===
\begin{frmae}

    

    
\end{frmae}


%===

%===

%===

%===

%===



\end{document}
