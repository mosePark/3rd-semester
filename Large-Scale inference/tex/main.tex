% ref link : https://ko.overleaf.com/latex/templates/texas-a-and-m-university-metropolis-beamer-theme/spvwhrxxbyqb
% This presentation uses a template designed by Dan Drennan, licensed under CC BY 4.0.


\documentclass[10pt]{beamer}

\usetheme[progressbar=frametitle]{metropolis}
\usecolortheme{aggie}

\usepackage{appendixnumberbeamer}
\usepackage{tcolorbox} % 내 설정
\usepackage{booktabs}
\usepackage[scale=2]{ccicons}


\usepackage{pgfplots}
\usepgfplotslibrary{dateplot}

\usepackage{xspace}
\newcommand{\themename}{\textbf{\textsc{metropolis}}\xspace}
\newcommand{\printfnsymbol}[1]{%
  \textsuperscript{\@fnsymbol{#1}}%
}
\makeatother
\newenvironment{callig}{\fontfamily{qcr}\selectfont}{}
\newcommand{\textcallig}[1]{{\callig#1}}
\newcommand{\f}{v}
\newcommand{\g}{g}
\newcommand{\hY}{\hat{y}}
\newcommand{\x}{x}
\newcommand{\z}{z}
\newcommand{\I}{\delta}
%\newcommand{\R}{\mathbb{R}}
\newcommand{\expd}{\text{\textcallig{p}}}
\newcommand{\ex}{\Expl}
\newcommand{\prodd}{\Upsilon}
\def\Expl{\mathcal{E}}
\newcommand{\e}{\mathbf{e}}
\newcommand{\m}{m}
\newcommand{\M}{M}
\newcommand{\bfalpha}{\mathbf{\alpha}}
\newcommand{\veck}{\mathbf{k}}
\newcommand*{\medcup}{\mathbin{\scalebox{1.5}{\ensuremath{\cup}}}}
\newcommand{\mubar}[1]{\bar{\mu}_{#1}}
\newcommand{\setlessell}{\mathcal{S}_\ell }
\newcommand{\bfb}{\mathbf{b} }
\DeclareMathOperator{\spann}{span}
\DeclareMathOperator{\calA}{\mathcal{A}}
\newcommand*{\defeq}{\stackrel{\text{def}}{=}}
\DeclareMathOperator*{\argmax}{arg\,max}
\DeclareMathOperator*{\argmin}{arg\,min}
\newcommand{\norm}[1]{\left\lVert#1\right\rVert}


% start == Title Page %%%
\title{Faith-SHAP}
\subtitle{Faith-Shap: The Faithful Shapley Interaction Index}
% \date{\today}
\date{May 14, 2024}
\author{mose Park}
\institute{    Department of Statistical Data Science \\
    University of Seoul}
% \titlegraphic{\hfill\includegraphics[height=1.5cm]{logo.pdf}}
% end == Title Page %%%

%%% def axiom box
% 사용자 정의 tcolorbox 환경을 만듭니다.
\newtcolorbox{myaxiombox}[2][]{%
    title=#2,
    colback=white,
    colframe=red!50!black,
    coltitle=white,
    fonttitle=\bfseries,
    rounded corners,
    boxsep=5pt,
    boxrule=1pt,
    #1 % 추가 옵션을 위한 공간
}

\newtcolorbox{mydefbox}[2][]{%
    title=#2,
    colback=white,
    colframe=red!50!gray, % 여기에 오타 수정
    coltitle=white,
    fonttitle=\bfseries,
    rounded corners,
    boxsep=5pt,
    boxrule=1pt,
    #1 % 추가 옵션을 위한 공간
}


%%%
\begin{document}

\maketitle

%===
\begin{frame}{Table of contents}
  \setbeamertemplate{section in toc}[sections numbered]
  \tableofcontents%[hideallsubsections]
\end{frame}
%===

%===
\begin{frame}{Overview}
    \begin{figure}[h]
    \centering
    \includegraphics[width=\textwidth]{fig/overview.png}
    \caption{overview}
    \end{figure}
\end{frame}
%===


\section[Intro]{Introduction}


%===
\begin{frame}{Coalition game}

\begin{figure}[h]
\centering
\includegraphics[width=0.8\textwidth]{fig/coalition-game.png}
\caption{coalition game}
\label{fig:coalition-game}
\end{figure}
  
\end{frame}


\def\model{f}
\def\X{\mathcal{X}}
\def\R{\mathbb{R}}
%===
\begin{frame}{Notation}
\begin{itemize}
    \itemsep1.5em % 더 큰 간격으로 각 항목 사이의 공간을 일정하게 조정
    \item $\model : \X \mapsto \mathbb{R}$, where $\X \subseteq \R^d$: A black-box model
    \item $\f_{\x} : 2^d \rightarrow \mathbb{R}$: A set function
    \item $\ell \in [d]$: The importance of interactions between features
    \item $\ex$: The importance function
    \item $\ex(\f, \ell) = (\ex_{S}(\f, \ell))_{S \in \mathcal{S}_\ell} \in \R^{d_\ell}$
    \begin{itemize}
        \vspace{0.3cm}
        \itemsep1.2em
        \item $\mathcal{S}_\ell$: Each coalition $S \subseteq [d]$ where $0 \leq |S| \leq \ell$
        \item $d_{\ell} \defeq \sum_{j=0}^{\ell} \binom{d}{j}$: The number of possible coalitions at level $\ell$
        \item $\ex(\f, \ell) \in \R^{d_\ell}$: The importance quantity at level $\ell$
    \end{itemize}
\end{itemize}
\end{frame}
%===
\section[Axioms]{Axioms}

\begin{frame}{Axioms}
    \begin{myaxiombox}{Interaction Linearity}
    For any maximum interaction order $\ell \in [d]$, and for any two set functions $\f_1$ and $\f_2$, and any two scalars $\alpha_1, \alpha_2 \in \mathbb{R}$, the interaction index satisfies: $\ex(\alpha_1 \f_1+ \alpha_2 \f_2,\ell) = \alpha_1 \ex(\f_1,\ell) + \alpha_2 \ex(\f_2,\ell)$.
    \end{myaxiombox}
\end{frame}
%===
\begin{frame}
    \begin{myaxiombox}{Interaction Symmetry}
        For any maximum interaction order $\ell \in [d]$, and for any set function $\f:2^d \mapsto \R$ that is symmetric with respect to elements $i, j \in [d]$, so that
$\f(S \! \cup  i) = \f(S \! \cup  j) \!$ for any $S \subseteq [d] \backslash \{i,j\}$, the interaction index satisfies: $\ex_{T \cup i}(\f,\ell) = \ex_{ T \cup j}(\f,\ell)$ for any $T \subseteq [d] \backslash \{i,j\}$ with $|T| < \ell$.
    \end{myaxiombox}

\end{frame}


%===
\begin{frame}
    \begin{myaxiombox}{Interaction Dummy}
        For any maximum interaction order $\ell \in [d]$, and for any set function $\f:2^d \mapsto \R$ such that $\f(S \cup i) = \f(S)$ for some $i \in [d]$ and for all $ S \subseteq [d] \backslash \{i\}$, the interaction index satisfies:  $\ex_{T}(\f, \ell) = 0$ for all $T \in \mathcal{S}_\ell$ with $i \in T$.
    \end{myaxiombox}
\end{frame}
%===
\begin{frame}
    \begin{myaxiombox}{Interaction Efficiency}
        For any maximum interaction order $\ell \in [d]$, and for any set function $\f:2^d \rightarrow \R$, the interaction index satisfies: $\sum_{S \in \mathcal{S}_\ell\backslash\text{\O}} \ex_S(\f,\ell) = \f([d]) - \f(\text{\O})$ and $\ex_{\text{\O}}(\f,\ell) = \f(\text{\O})$.
    \end{myaxiombox}
\end{frame}
%===
\section[Interaction index]{Interaction index}
%===
\begin{frame}{Interaction index}
    \begin{mydefbox}{Shapley Interaction Index}
        \begin{equation*}
            \label{eqn:closed_form_shap_inter}
            \ex_S^{\text{Shap}}(\f, \ell) =  \sum_{T \subseteq [d]/ S}  \frac{|T|! (d-|S|-|T|)!}{(d-|S|+1)!} \Delta_S(\f(T)),\ \ \  
            \forall S \in \setlessell.
            \tag{5}
        \end{equation*}
    \end{mydefbox}
\end{frame}
%===
\begin{frame}
    \begin{mydefbox}{Banzhaf Interaction Index}
        \begin{equation*}
            \label{eqn:closed_form_bzf_inter}
            \ex^{\text{Bzf}}_S(\f,\ell) =  \sum_{T \subseteq [d]/S}  \frac{1}{2^{d-|S|}} \Delta_S(\f(T)),\ \ \  
            \forall  S \in \setlessell.
            \tag{6}
        \end{equation*}
    \end{mydefbox}
\end{frame}
%===
\begin{frame}
    \begin{mydefbox}{Shapley Taylor Interaction Index}
        \begin{equation*}
            \label{eqn:def_taylor_shap}
            \ex_S^{\text{Taylor}}(\f,\ell) = 
            \begin{cases}
            \Delta_S(\f(\text{\O})) & \text{, if } |S| < \ell. \\
            \sum_{T \subseteq [d]/ S}  \frac{|T|!(d-|T|-1)!|S|}{d!} \Delta_S(\f(T)) & \text{, if } |S| = \ell. \\
            \end{cases}
            \tag{7}
        \end{equation*}
    \end{mydefbox}
\end{frame}
%===
\begin{frame}{Faith-Interaction Indices}
    \textbf{Singleton Attributions to Interaction Indices} : \\
    Let value function with $S$, \[\f(S) \approx \sum_{T \subseteq S, |T| \leq \ell} \Expl_T(\f,\ell),\; \forall S \subseteq [d].\]
    And then, we consider the following weghted regression problem,
    \begin{equation*}
    \label{eqn:weighted_regression}
    \ex(\f, \ell) \ = \argmin_{\Expl \subseteq \mathbb{R}^{d_\ell} } 
    \sum_{S \subseteq [d]}  \mu(S) \left( \f(S) - \sum_{T \subseteq S , |T| \leq \ell}\Expl_T(\f,\ell) \right)^2,
    \tag{9}
    \end{equation*}
    % weighting fuction이 무엇인지 설명 넣어주고 제약이 들어간 최적화 문제로 다시 설명할 것. (다음 페이지)
\end{frame}
%===
\begin{frame}
    Finally, we can induce as solving the constrained problem:
    \begin{align*}
    \ex(\f, \ell) \ 
    &= \argmin_{\Expl \subseteq \mathbb{R}^{d_\ell} } \sum_{S \subseteq [d]\,:\, \mu(S) < \infty}  \mu(S) \left( \f(S) - \sum_{T \subseteq S , |T| \leq \ell}\Expl_T(\f,\ell) \right)^2 \nonumber \\
    \text{s.t.} \ & \f(S) = \sum_{T \subseteq S , |T| \leq \ell}\Expl_T(\f,\ell), \;\;\forall S :\, \mu(S) = \infty.
    \label{eqn:constrained_weighted_regresion}
    \tag{10}
    \end{align*}
\end{frame}
%===
\begin{frame}
    
\end{frame}
%===
\begin{frame}
    
\end{frame}
%===
\begin{frame}
    
\end{frame}
%===
\begin{frame}
    
\end{frame}
%===
\begin{frame}
    
\end{frame}







%===
\end{document}



